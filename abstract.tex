\chapter*{Abstract}
\addcontentsline{toc}{chapter}{Abstract}
Fengycin is a lipopeptide synthesized by a bacteria in response to fungal attacks. 
They have a cyclic peptide head composed of ten amino acids and a long hydrophobic 
tail. %Fengycins in water form micelles due to their amphiphilic nature. 
Aggregate size of fengycins on the membrane surface after binding was found to be critical for the estimating the rate of membrane leakage caused by fengycins 
in previous fluorescence lifetime experiments and coarsed-grain simulations.
In my thesis
I determined how fengycin selectively prefers to aggregate in one membrane over the
other depending on the membrane\'s composition. Using all-atom molecular dynamics simulations, I found that fengycin's peptide head favorably 
interacts with phospholipid heads and that reduces peptide-peptide interactions 
which leads to less aggregation. On the other hand when 
phospholipid heads have groups which are not favorably interacting with fengycin's peptide 
part, fengycins tend to form long-lived aggregates.
Thus in model bacterial membranes composed of phospholipids POPE and POPG
fengycins form smaller aggregates due to increased interactions between basic rsidues in peptides
and the ammonium group in POPE. On the other hand when the phospholipids POPC in model fungal membranes 
failed to form electrostatic interactions with fengycins' head group
then the peptides\' hydrophobic interactions predominate leading to larger aggregates.

We also used coarsed-grain molecular dynamics simulations with weighted ensemble 
sampling to determine the effects of cholesterol on fengycin aggregation on the 
membrane surface. Previous experiments suggested that when cholesterol is present
fengycins take more time to cause membrane leakage. This can be attributed to 
slower binding of fengycins to the membrane or less fengycin aggregation 
on the membrane surface after binding. 
Here, we only looked at the aggregation propensity
of fengycins in the presence of cholesterol after binding.
After cumulative simulation time of \~0.9 ms we found that fengycins don't interact 
preferentially with cholesterols.
But we found that in the presence of cholesterol , membrane bending is reduced
in the regions close to fengycin aggregates.
 This can be attributed to the cholesterol's inherent tendencies to pack
 and order lipids. And hence this might be the reason why the fengycin's rate of membrane leakage is reduced.
 
%%%write here
