\chapter{Introduction}
\label{chap:introduction}
\addcontentsline{toc}{chapter}{Biographical Sketch}
\section{Human population and the need for improved agriculture}
In 2019 the human population has reached 7.7 billion. And this number will continue to increase.
According to United nations the human population will reach 9.8 billion by 2050.
This growing population requires minimum food and clothing.
Both of these resources can be increased with a better and improved agricultural practices. 
Even if we increase the cultivable land the amount of crops destroyed by bacteria, fungi and
pests is a considerable amount. According to Berber et al, as the temperature increases the problem of microorganisms and pests is going to grow.\cite{Gurr2013} 
In fact every year, crops are lost due to various diseases and pests is almost 10-15\%.
One of the ways to tackle this issue is inventing better and improved antibiotics. 

\subsection{Antibiotics: Pros and Cons and it's rich history}
Antibiotics have been used by mankind since 300-350BC. Evidence of which has been found in the 
human remains from that era and even in the Roman period too. \cite{Armelagos2010, 
Villanueva1980, Anderson1989} 
Even before western medicine caught up to antibiotics, traditional forms 
of medicine that were essentially antibiotics have always been present 
in various cultures.\cite{Aminov2010}The most famous example of this is 
the discovery of  artemisinin also known as 'qinghaosu' in china where 
this has been used as a medicine for hundreds of years. Artemesinin was 
extracted from  Artemisia plants in 1970 as a potent antimalarial 
drug.\cite{Su2009}

The modern antibiotic era started with Paul Ehrlich and Alexander Fleming. 
Ehrlich observed that certain dyes like aniline, stain only certain bacteria and not all. 
From this he concluded that targeted chemical compounds can be found that 
kills only disease causing organism and not the healthy cells. 
Their target disease was syphilis which was endemic and a need for less drastic medication
was necessary. In modern drug discovery terms, what we call high thoroughput screening 
they went through hundreds of compounds before stumbling upon 
Neosalvarsan.\cite{Hata1910}
After this came the era of antibiotics where different chemical compound classes were 
found which were effective against various bacteria and fungus, namely penicillin which is effective against broad spectrum of bacteria.\cite{Fleming1929}
But these drugs come with a huge price. Microorganisms can develop resistance towards 
these compounds quickly. In fact we now have evolved microorganisms called "superbugs" that 
are resistant to most antibiotics. Each year thousands of patients die in 
hospitals in developed countries after contracting multi-drug resistant 
drug.\cite{@roderick2007}
Soon after penicillin came into market, patients started showing resistance towards 
penicillin due to $\beta$-lactamase.
%%%%%Add citation%%%%%

%%%Write a little about the history of lactamases


    -- Membranes 
        -- How did it evolve from bacteria to eukaryotes? 
        -- Phospholipids 
        -- Sterols 
    -- Antimicrobial peptides 
        -- Margainins first observed 
        -- Several on our skin 
        -- D-amino acids 
        -- Targets cell membrane 
    -- Fengycin
        --synthesized where 
        -- functions 
        -- selectivity 
        -- Serenade 

\subsection{What are biological membranes?}
\label{s:membrane_intro}
-- Membranes definition 
-- Fascinating things about membrane 
-- Hydrophobic and hydrohillic parts 
-- How does membrane affect protein function? 
    -- Lipids can act as a ligand (Leslie's paper) 
    -- Electrostatics 
    -- Membrane curvature 

\subsection{What is the problem?}
Experiments done on Fengycin 
-- ITC 
-- Fluorescence lifetime 
-- Atomic spectroscopy 
-- Total internal reflection 
-- TEM 


\subsection{Molecular dynamics simulations}
-- How does it work 
    -- equations of motion
    -- force field terms 
-- Models used: 
    -- All atom models 
    -- Coarse-grain models 
-- System setup 
    -- how did we build the fengycins bound to membrane 
    -- lipid positions around protein 
    -- water concentration 
    -- OMG 
    -- Insane 
-- Analysis 
-- Temperature and pressure control 
-- Free energy methods and enhanced sampling 
